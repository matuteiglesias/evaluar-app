

 \item En el \emph{problema de gestión de proyectos} tenemos un proyecto que se divide en $n$ etapas $v_1, \ldots, v_n$.  Cada etapa $v_i$ consume un tiempo $t_i \geq 0$. Para poder empezar una etapa $v_i$ se requiere que primero se hayan terminado un conjunto $N(v_i)$ de etapas $v_j$ tales que $j < i$.  Por simplicidad, la etapa $v_1$ se usa como indicador de inicio del proyecto y, por lo tanto, consume un tiempo $t_1 = 0$ y es requerida por todas las otras etapas.  Análogamente, la etapa $v_n$ indica el final del proyecto por lo que consume tiempo $t_n = 0$ y requiere la finalización del resto de las etapas.  Una etapa es \emph{crítica} cuando cualquier atraso en la misma provoca un retraso en la finalización del proyecto.  Modelar el problema de encontrar todas las etapas críticas de un proyecto como un problema de camino mínimo e indicar qué algoritmo usaría para resolverlo.  El mejor algoritmo que conocemos toma tiempo lineal en la cantidad de datos necesarios para describir un proyecto.


