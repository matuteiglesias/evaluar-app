El entrenador de la selección argentina de fútbol, el gran Diego Maradona, va a probar una nueva formación este año. La formación describe cómo se posicionan los jugadores en el campo. En lugar de las convencionales 4-4-2 o 4-3-3, ha optado por una 5-5. Esto significa que hay 5 atacantes y 5 defensores. 

La Asociacion de Fútbol Argentina (AFA) te ha contratado para escribir un código que les ayudará a determinar qué jugadores deberían tomar las posiciones de ataque/defensa. Maradona te ha dado una lista con los nombres de los 10 jugadores que saldrán al campo. También se proporciona la habilidad de ataque y la habilidad defensiva de cada jugador. Tu trabajo es determinar qué 5 jugadores deberían tomar las posiciones de ataque y cuáles 5 las de defensa. 

Las reglas que se deben seguir para tomar la decisión son: 

• La suma de las habilidades de ataque de los 5 atacantes necesita maximizarse 

• Si hay más de una combinación, maximizar la suma de las habilidades defensivas de los 5 defensores 

• Si aún hay más de una combinación, elegir a los atacantes que vengan primero lexicográficamente. 

Entrada 

La primera línea de entrada contiene un entero T (T < 50) que indica el número de casos de prueba. Cada caso contiene exactamente 10 líneas. La i-ésima línea contiene el nombre del i-ésimo jugador seguido por su habilidad de ataque y defensiva respectivamente. La longitud del nombre de un jugador es de un máximo de 20 y consiste solo en letras minúsculas. Las habilidades de ataque/defensa son enteros en el rango [0, 99]. 

Salida 

La salida de cada caso contiene tres líneas. La primera línea es el número de caso empezando desde 1. La siguiente línea contiene el nombre de los 5 atacantes en el formato ‘(A1, A2, A3, A4, A5)’ donde Ai es el nombre de un atacante. La siguiente línea contiene el nombre de los 5 defensores en el mismo formato. Los nombres de los atacantes y defensores se deben imprimir en orden lexicográfico ascendente. Mira el ejemplo para más detalles.
