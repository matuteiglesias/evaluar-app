

\item Dado un digrafo completo y pesado $D$ el \emph{problema de viajante de comercio} (TSP por sus siglas en inglés: \emph{traveling salesman problem}) consiste en encontrar un ciclo que recorra todos los vértices de $D$ y tenga costo mínimo. Queremos resolver el caso particular de TSP en el cual $|V(D)| = 2n$ y sabemos en qué orden deben recorrerse los nodos ``pares''. Es decir, además de $D$, el input contiene una secuencia $w_2, w_4, \ldots, w_{2n}$ de vértices; el output debe ser un ciclo $v_1, v_2, \ldots, v_{2n}$ tal que $v_{2i} = w_{2i}$ para todo $i = 1, \ldots, n$.

\begin{enumerate}[label=$\alph*)$,ref=$\alph*)$]
 \item Modelar el TSP como un problema de matching bipartito de peso mínimo en grafo $G$.
 \item Dar una interpretación a cada matching de $G$ como representante de un ciclo de $D$.
 \item Demostrar que el modelo es correcto.
 \item Determinar la complejidad de resolver el modelo resultante con el algoritmo del Ejercicio~\ref{ej:matching}.
\end{enumerate}

