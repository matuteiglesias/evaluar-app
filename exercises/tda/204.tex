
Calcule la complejidad de un algoritmo que utiliza $T(n)$ pasos para una
entrada de tamaño $n$, donde $T$ cumple:


<div style="column-count: 3;">

\item $T(n) = T(n-2) + 5$
\item $T(n) = T(n-1) + n$
\item $T(n) = T(n-1) + \sqrt{n}$
\item $T(n) = T(n-1) + n^2$
\item $T(n) = 2 T(n-1)$
\item $T(n) = T(n/2) + n$
\item $T(n) = T(n/2) + \sqrt{n}$
\item $T(n) = T(n/2) + n^2$
\item $T(n) = 2 T(n-4)$
\item $T(n) = 2 T(n/2) + \log n$
\item $T(n) = 3 T(n/4)$
\item $T(n) = 3 T(n/4) + n$

</div>


Intentar estimar la complejidad para cada ítem directamente y luego calcularla
utilizando el teorema maestro de ser posible. Para simplificar los cálculos se
puede asumir que $n$ es potencia o múltiplo de 2 o de 4 según sea conveniente.

