
 \item Un \emph{sistema de restricciones de diferencias (SRD)} es un sistema ${\cal S}$ que tiene $m$ inecuaciones y $n$ incógnitas $x_1, \ldots, x_n$.  Cada inecuación es de la forma $x_i - x_j \leq c_{ij}$ para una constante $c_{ij} \in \mathbb{R}$; por cada par $i,j$ existe a lo sumo una inecuación (por qué?).  Para cada SRD ${\cal S}$ se puede definir un digrafo pesado $D({\cal S})$ que tiene un vértice $v_i$ por cada incógnita $x_i$ de forma tal que $v_j \to v_i$ es una arista de peso $c_{ij}$ cuando $x_i - x_j \leq c_{ij}$ es una inecuación de ${\cal S}$.  Asimismo, ${\cal S}$ tiene un vértice $v_0$ y una arista $v_0 \to v_i$ de peso $0$ para todo $1 \leq i \leq n$. \label{ej:SRD}

 \begin{enumerate}[label=$\alph*$.,ref=$\alph*$]
  \item Demostrar que si $D({\cal S})$ tiene un ciclo de peso negativo, entonces ${\cal S}$ no tiene solución.
  \item Demostrar que si $D({\cal S})$ no tiene ciclos de peso negativo, entonces $\{x_i = d(v_0, v_i) \mid 1 \leq i \leq n\}$ es una solución de $D({\cal S})$.  Acá $d(v_0, v_i)$ es la distancia desde $v_0$ a $v_i$ en $D({\cal S})$.
  \item A partir de los incisos anteriores, proponer un algoritmo que permita resolver cualquier SRD.  En caso de no existir solución, el algoritmo debe mostrar un conjunto de inecuaciones contradictorias entre sí.
 \end{enumerate}


