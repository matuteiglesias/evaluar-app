
 \item Sea $S$ una cadena con $n$ paréntesis que abren y $n$ paréntesis que cierran.  Dada una longitud $\ell$ impar, decimos que $s\colon \{1,\ldots, n\} \to \mathbb{N}$ es un \emph{$\ell$-posicionamiento uniforme} de $S$ si $s(i)$ es par y al escribir el $i$-ésimo paréntesis que abre en $s(i)$ y el $i$-ésimo paréntesis que cierra en $s(i)+\ell$, $1 \leq i \leq n$, se obtiene una escritura válida de $S$. Por ejemplo, si $S = (\,(\,(\,)\,(\,(\,)\,(\,)\,)\,)\,(\,)\,)$ y $\ell = 15$, entonces $s(1) = 0$, $s(2) = 6$, $s(3) = 10$, $s(4) = 16$, $s(5) = 20$, $s(6) = 22$ y $s(7) = 36$ es un $15$-posicionamiento uniforme de $S$.  Definir un SRD que permita resolver el problema de determinar si una cadena dada $S$ tiene un $\ell$-posicionamiento uniforme cuando $\ell>0$ impar también es dado.  El mejor SRD que conocemos tiene $O(n)$ inecuaciones y, por lo tanto, permite resolver el problema en $O(n^2)$ de aplicando el algoritmo del ejercicio anterior.


