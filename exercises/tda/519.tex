

 \item Solo para este ejercicio, la clase de los grafos \emph{densos} está formada por todos los grafos con $\Omega(n^2)$ aristas, mientras que la clase de los grafos \emph{ralos} está formada por todos los grafos con $O(n)$ aristas.  Justificar qué algoritmo de camino mínimo conviene usar para cada uno de los siguientes problemas, explicitando su implementación:
 \begin{enumerate}[label=$\alph*$.,ref=$\alph*$]
   \item Encontrar un camino mínimo de uno a todos en un grafo ralo (resp.\ denso) cuyos pesos son todos iguales y no negativos.
   \item Encontrar un camino mínimo de todos a todos en un grafo ralo (resp.\ denso) cuyos pesos son todos iguales y no negativos.
   \item Encontrar un camino mínimo de uno a todos en un grafo ralo (resp.\ denso) cuyos pesos son no negativos.
   \item Encontrar un camino mínimo de todos a todos en un grafo ralo (resp.\ denso) cuyos pesos son no negativos.
   \item Determinar si un grafo ralo (resp.\ denso) tiene ciclos de peso negativo; no suponer que el grafo es conexo.
   \item Encontrar un camino mínimo de uno a todos en un grafo ralo (resp.\ denso).
   \item Encontrar un camino mínimo de todos a todos en un grafo ralo (resp.\ denso).
 \end{enumerate}


