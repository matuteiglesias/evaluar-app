\label{mali}
Escriba un algoritmo con dividir y conquistar que determine si un arreglo de tamaño potencia de 2 es \emph{más a la izquierda}, donde ``más a la izquierda''\ significa que:
%\begin{itemize}
\item La suma de los elementos de la mitad izquierda superan los de la mitad derecha.
\item Cada una de las mitades es a su vez ``más a la izquierda''.
%\end{itemize}
Por ejemplo, el arreglo [8, 6, 7, 4, 5, 1, 3, 2] es ``más a la izquierda'', pero [8, 4, 7, 6, 5, 1, 3, 2] no lo es.

Intente que su solución aproveche la técnica de modo que
complejidad del algoritmo sea estrictamente menor a $O(n^2)$.


%Suponer por simplicidad que la longitud del arreglo es potencia de 2.

