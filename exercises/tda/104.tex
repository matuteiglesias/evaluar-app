

\item Dada una matriz $D$ de $n \times n$ números naturales, queremos encontrar una permutación $\pi$\footnote{Una permutacion de un conjunto finito $X$ es simplemente una función biyectiva de $X$ en $X$.} de $\{1, \ldots, n\}$ que minimice $D_{\pi(n)\pi(1)} + \sum_{i=1}^{n-1} D_{\pi(i)\pi(i+1)}$.  Por ejemplo, si

\[
    D = \begin{pmatrix}
          0  &  1 & 10 & 10 \\
          10 &  0 &  3 & 15 \\
          21 &  17 &  0 & 2 \\
           3 &  22 & 30 & 0
        \end{pmatrix}
\]


entonces $\pi(i) = i$ es una solución optima.

\begin{enumerate}
 \item Diseñar un algoritmo de \emph{backtracking} para resolver el problema, indicando claramente cómo se codifica una solución candidata, cuáles soluciones son válidas y qué valor tienen, qué es una solución parcial y cómo se extiende cada solución parcial.
 \item Calcular la complejidad temporal y espacial del mismo.
 \item Proponer una poda por optimalidad y mostrar que es correcta.
\end{enumerate}

\end{enumerate}


