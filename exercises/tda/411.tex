
  
  \item\Obligatorio Una empresa de comunicaciones modela su red usando un grafo $G$ donde cada arista tiene una capacidad positiva que representa su \emph{ancho de banda}.  El \emph{ancho de banda} de la red es el máximo $k$ tal que $G_k$ es conexo, donde $G_k$ es el subgrafo generador de $G$ que se obtiene de eliminar las aristas de peso menor a $k$ (Figura~\ref{fig:ancho-banda}).
  
 \begin{enumerate}[label=$\alph*)$,ref=$\alph*)$]
  \item Proponer un algoritmo eficiente para determinar el ancho de banda de una red dada.
 \end{enumerate}
 
 La empresa está dispuesta a hacer una inversión que consiste en actualizar algunos enlaces (aristas) a un ancho de banda que, para la tecnología existente, es virtualmente infinito.  Antes de decidir la inversión, quieren determinar cuál es el ancho de banda que se podría obtener si se reemplazan $i$ aristas para todo $0 \leq i < n$.
 
 \begin{enumerate}[resume]
  \item Proponer un algoritmo que dado $G$ determine el vector $a_0, \ldots, a_{n-1}$ tal que $a_i$ es el ancho de banda máximo que se puede obtener si se reemplazan $i$ aristas de $G$.
 \end{enumerate}


% FIGURA
