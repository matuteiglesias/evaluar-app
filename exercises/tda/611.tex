
 
 \item Una de las aficiones de Carle en su juventud fue la colección de figuritas en el colegio.  Junto a sus compañeres compraban paquetes de figuritas de ``Italia 90'' para conocer a las estrellas del momento.  Cada paquete traía cuatro figuritas a priori desconocidas, razón por la cual Carle y sus compañeres tenían figuritas repetidas después de algunas compras.  Para completar el álbum más rápidamente, Carle y sus compañeres intercambiaban figuritas a través del protocolo ``late-nola''.  Este protocolo consiste en que cada una de dos personas intercambian una figurita que elles tienen repetida por una que no poseen aún.  Siendo tan inteligente, Carle pronto se dio cuenta que le podía convenir intercambiar algunas de sus figuritas por otras que ya tenía, a fin de intercambiar estas últimas.  De esta forma, si Carle ya tenía copias de una figurita, igualmente podía conseguir copias adicionales para intercambiar con otres compañeres que no tuvieran la figurita.  
 \begin{enumerate}[label=$\alph*)$,ref=$\alph*)$]
  \item Proponer un modelo de flujo máximo para maximizar la cantidad de figuritas no repetidas que Carle puede obtener a través del intercambio con compañeres, teniendo en cuenta las siguientes observaciones:
 %\begin{itemize}
  \item Carle conoce todas las figuritas repetidas (y la cantidad de repeticiones) de cada compañere.
  \item Todes les compañeres intercambian primero con Carle, antes de intercambiar entre elles.
  \item Todes les compañeres utilizan el protocolo ``late-nola'' para intercambiar con Carle, mientras que Carle ya sabe que le podría convenir obtener figuritas que ya tiene.
 %\end{itemize}
  \item Dar una interpretación a cada unidad de flujo y cada restricción de capacidad.
  \item Determinar la complejidad de resolver el modelo resultante con el algoritmo de Edmonds y Karp.
 \end{enumerate} 
\end{enumerate}


