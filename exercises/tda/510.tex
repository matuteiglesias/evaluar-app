 
 \item Tenemos a $n$ clientes de un supermercado $\{c_1, c_2, ..., c_n\}$ y queremos asignarle a cada uno, una caja para hacer fila. Las cajas están ordenadas en una línea y numeradas de izquierda a derecha de la $1$ a la $M$ y se encuentran separadas por pasillos.  Durante el proceso de asignación algunos clientes se pelean entre sí y son separados por seguridad. Si dos clientes $c_i$ y $c_j$ pelean, los guardias les dicen que tienen que ponerse en filas distintas que se encuentren separadas por $K_{ij} > 0$ pasillos intermedios, para que no se vuelvan a pelear. Notar que cuando seguridad separa una pelea naturalmente hay un cliente que queda más a la izquierda (cerca de la caja $1$) y el otro más a la derecha (cerca de la caja $M$). Con la restricción de no volver a acercarse, ese orden ya no puede cambiar. A su vez hay pares de clientes $c_k$ y $c_m$ que son amigos y no queremos que haya más que $L_{km} = L_{mk} \geq 0$ pasillos intermedios entre las filas de $c_k$ y $c_m$. ¿será posible asignarlos a todos?


 \begin{enumerate}[label=$\alph*$.,ref=$\alph*$]
  \item Modelar el problema utilizando un sistema de resticciones de diferencias (no olviden justificar).
  \item Proponer un algoritmo polinomial que lo resuelva.
  \item ¿Qué complejidad tiene el algoritmo propuesto?  Para la respuesta, tener en cuenta la cantidades $m_1$ y $m_2$ de amistades y peleas, respectivamente.
 \end{enumerate}
 
 \textbf{Nota:} $K_{ij}$ de alguna manera captura la intensidad de la pelea y $L_{ij}$ captura (inversamente) la intensidad de la amistad. Es posible que dos amigos se peleen y en ese caso hay que cumplir las dos condiciones. Si eso pasa solo puede haber soluciones si $K_{ij} \leq L_{ij}$. Para todo par de clientes sabemos si son amigos o si se pelearon, la intensidad de cada relación.  Además, para aquellos clientes que se pelearon, conocemos cuál cliente quedó a la izquierda y cuál a la derecha.
 \newline
 \textbf{Ayuda:} Si tenemos $n$ variables $x_i$ en un SRD y queremos acotarlas entre $A$ y $B$  ($x_i \in [A, B]$) podemos agregar una variable auxiliar $z$, sumar restricciones del tipo $A \leq x_i - z \leq B$ y luego correr la solución para que $z$ sea $0$.

