
Se tiene un arreglo de n\'umeros naturales $A$. Adem\'as se cuenta con
estructuras adicionales sobre el arreglo que proveen la
funci\'on
\begin{center}
aparece?(\textbf{in} $A$: arreglo(nat), \textbf{in} $i$:nat, \textbf{in} $j$:nat, \textbf{in} $e$: nat) $\to$ bool
\end{center}
que dado el arreglo $A$, \'indices $i,j$ y un natural $e$, devuelve \textit{true}
si y s\'olo si $e = A[k]$ para alg\'un $k$ tal que $i \le k \le j$. Adem\'as se
sabe que aparece? toma tiempo $O(\sqrt{j-i+1})$, es decir, la raiz cuadrada del
tama\~no del intervalo de b\'usqueda.

Se desea encontrar un algoritmo sublineal 
que encuentra el \'indice de un elemento $e$ en el arreglo $A$, asumiendo que
tal elemento existe en el arreglo.
El resultado de la funci\'on es justamente el \'indice $i$ tal que $A[i] = e$.

\begin{enumerate}
\item[a)] Implementar la funci\'on
\begin{center}
ubicar?(\textbf{in} $A$: arreglo(nat), \textbf{in} $e$: nat) $\to$ nat
\end{center}
que resuelve el problema planteado. 
La función debe ser de tiempo estrictamente menor a $O(n)$, dónde
$n = \tam(A) = \tam(B)$ (formalmente, la complejidad del algoritmo \textbf{no}
debe pertenecer a $\Omega(n)$)
\item[b)] Calcular y justificar la complejidad del algoritmo propuesto. 
\end{enumerate}

