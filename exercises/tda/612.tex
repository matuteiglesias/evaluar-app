
\item Una \emph{red con costos} es una red en la que cada arista $e$ tiene una capacidad $c(e)$ y un costo $q(e) \geq 0$.  Dada una red con costos $N$, el \emph{problema de flujo máximo con costo mínimo} consiste en encontrar el flujo máximo $f$ que minimice $\sum_{e \in E(N)} f(e) * q(e)$.  Demostrar que el algoritmo de Ford y Fulkerson, en el que el camino de aumento elegido tiene costo mínimo, encuentra un flujo máximo de costo mínimo.  Determinar qué algortimo se utilizar para elegir el camino de aumento y calcular la complejidad del algoritmo resultante (tener en cuenta que el algoritmo requiere a lo sumo $O(nU)$ iteraciones, donde $U=\max_{e\in E(N)}c(e)$). \label{ej:fmcm}


