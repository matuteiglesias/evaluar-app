
 \item Dado un grafo $G$ con capacidades en sus aristas, el \emph{ancho de banda} $\bwd_G(C)$ de un camino $C$ es la mínimo de entre las capacidad de las aristas del camino (Figura~\ref{fig:ancho-banda}).  El \emph{ancho de banda} $\bwd_G(v,w)$ entre dos vértices $v$ y $w$ es el máximo entre los anchos de banda de los caminos que unen a $v$ y $w$ (Figura~\ref{fig:ancho-banda}).  Un árbol generador $T$ de $G$ es \emph{maximin} cuando $\bwd_T(v,w) = \bwd_G(v,w)$ para todo $v,w \in V(G)$.  Demostrar que $T$ es un árbol maximin de $G$ si y solo si $T$ es un árbol generador máximo de $G$.  Concluir que todo grafo conexo $G$ tiene un árbol maximin que puede ser computado con cualquier algoritmo para computar árboles generadores máximos. \textbf{Ayuda:} para la ida, tomar el AGM $T'$ que tenga más aristas en común con $T$ y suponer, para obtener una contradicción, que $T'$ tiene una arista $e'$ que no está en $T$.  Luego, buscar una arista $e$ en $T$ que no este en $T'$ tal que $(T' - e') + e$ sea también AGM para obtener la contradicción.  Para la vuelta, tomar $v$ y $w$ en el AGM $T'$ y considerar la arista $xy$ de peso mínimo en el único camino de $T'$ que los une.  Luego, mostrar que $xy$ tiene un peso al menos tan grande como cualquier otra arista que une las componentes conexas de $T' \setminus \{xy\}$ que contienen a $v$ y $w$.


