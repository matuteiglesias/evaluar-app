
 \item En el \emph{problema del vuelto} tenemos una cantidad ilimitada de monedas de distintos valores $w_1, \ldots, w_k$ y queremos dar un vuelto $v$ utilizando la menor cantidad de monedas posibles (ver Teórica 2).  Por ejemplo, si los valores son $w_1 = 1$, $w_2 = 5$, y $w_3 = 12$ y el vuelto es $v = 15$, entonces el resultado es $3$ ya que alcanza con dar $3$ monedas de $\$5$.  Modelar este problema como un problema de camino mínimo e indicar un algoritmo eficiente para resolverlo.  El algoritmo sobre el modelo debe tener complejidad $O(vk)$.  \textbf{Opcional:} discutir cómo se relaciona este modelo con el algoritmo de programación dinámica correspondiente.


