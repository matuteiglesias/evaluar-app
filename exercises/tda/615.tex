

\item Una \emph{red con demandas} es una red en la que cada arista $e$ tiene una capacidad $c(e)$ y una demanda $0 \leq d(e) \leq c(e)$.  Dada una red con demandas, el problema de flujo asociado consiste en encontrar un flujo válido $f$ tal que $d(e) \leq f(e) \leq c(e)$ para todo arco $e$.  Para resolver el problema de flujo en una red $N$ con demandas se puede resolver un problema de flujo en una red $N'$ sin demandas.  La red $N'$ se obtiene agregando una nueva fuente $s'$ y un nuevo sumidero $t'$ a $N$, una arista $t' \to s'$, y las aristas $s \to v$ y $v \to t$ para todo $v \in V(N)$.  La función de capacidad $c'$ de $N'$ es tal que:
\begin{enumerate}[label=$\roman*$.,ref=$\roman*$] 
 \item $c'(s' \to v) = \sum_{u \in V} d(u \to v)$ para todo $v \in V(N)$,
 \item $c'(v \to t') = \sum_{w \in V} d(v \to w)$ para todo $v \in V(N)$,
 \item $c'(u \to v) = c(u \to v) - d(u \to v)$ para todo arco $e \in E(N)$,
 \item $c'(t \to s) = \infty$.
\end{enumerate}
Demostrar que $N$ tiene un flujo factible si y sólo si el flujo máximo de $N'$ satura todas las aristas que salen de $s'$ (y todas las que entran a $t'$).  \textbf{Sugerencia:} muestre cómo se puede obtener un flujo factible de $N$ a partir de un flujo máximo de $N'$ y viceversa.  


