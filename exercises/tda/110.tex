

\item Tenemos cajas numeradas de $1$ a $N$, todas de iguales dimensiones. Queremos encontrar la máxima cantidad de cajas que pueden apilarse en una única pila cumpliendo que:
%\begin{itemize}
\item sólo puede haber una caja apoyada directamente sobre otra;
\item las cajas de la pila deben estar ordenadas crecientemente por número, de abajo para arriba;
\item cada caja $i$ tiene un peso $w_i$ y un soporte $s_i$, y el peso total de las cajas que están arriba de otra no debe exceder el soporte de esa otra.
%\end{itemize}
\par{Si tenemos los pesos $w = [19, 7, 5, 6, 1]$ y los soportes $s = [15, 13, 7, 8, 2]$ (la caja 1 tiene peso 19 y soporte 15, la caja 2 tiene peso 7 y soporte 13, etc.), entonces la respuesta es 4. Por ejemplo, pueden apilarse de la forma 1-2-3-5 o 1-2-4-5 (donde la izquierda es más abajo), entre otras opciones.}

\begin{enumerate}[label=$\alph*)$,ref=$\alph*)$]
\item Pensar la idea de un algoritmo de \textit{backtracking} (no hace falta escribirlo).
\item Escribir una formulación recursiva que sea la base de un algoritmo de PD. Explicar su semántica e indicar cuáles serían los parámetros para resolver el problema.
\item Diseñar un algoritmo de PD y dar su complejidad temporal y espacial auxiliar. Comparar cómo resultaría un enfoque \textit{top-down} con uno \textit{bottom-up}.
\item (Opcional) Formalizar el problema y demostrar que la función recursiva es correcta.
\end{enumerate}

