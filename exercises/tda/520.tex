
 
 \item %(Opcional\footnote{Este ejercicio es \textbf{difícil}; su propósito es repasar los invariantes de los algoritmos de camino mínimo en un ambiente práctico.  En caso de no resolverlo, se sugiere \textbf{fuertemente} repasar la teórica a conciencia teniendo en cuenta los invariantes de los algoritmos.}) 
 Se tiene un digrafo $D$ donde cada arista $v \to w$ representa un camino directo de una locación $v$ a otra $w$.  La arista $v \to w$ tiene un \emph{tiempo de viaje} $t(v \to w)$ que indica que si un vehículo parte de $v$ en el instante $t$, entonces llega a $w$ en el instante $t + t(v\to w)$.  Asimismo, $v \to w$ tiene un \emph{tiempo de apertura} $s(v \to w) \geq 0$ que indica que ningún vehículo puede empezar a cruzar $v \to w$ antes del instante $s(v \to w)$.  Dado un vértice $v$ y para todo $w \in V(G)$, queremos determinar el instante más temprano $d(w) \geq 0$ en que un vehículo puede llegar a $w$ si empieza su recorrido en $v$ (Figura~\ref{fig:viajes}).
 
 \begin{enumerate}[label=$\alph*$.,ref=$\alph*$]
  \item Suponiendo que $t(\cdot) \geq 0$, diseñar un algoritmo eficiente para el problema que esté basado en el algoritmo de Dijkstra; justificar su correctitud.  \textbf{Ayuda:} mantener como invariante una partición $V, W$ de $V(G)$ tal que: $d(x)$ es correcto para todo $x \in V$ y $d(y) \geq d(x)$ para todo $y \in W$.  Luego, determinar qué vértice de $W$ puede agregarse a $V$ satisfacendo el invariante.
  
  \item Sin suponer que $t(\cdot) \geq 0$ (i.e., algunas aristas ``vuelven el tiempo atras'') pero suponiendo que $t(C) \geq 0$ para todo ciclo $C$, diseñar un algoritmo eficiente para el problema que esté basado en el algoritmo de Bellman-Ford; justificar su correctitud.  \textbf{Ayuda:} definir una función recursiva $d^k(w)$ que para cada $w$ indique (una cota inferior de) qué tan temprano se puede llegar a $w$ si se permiten recorrer hasta $k$ aristas.
  
  \item Justifique brevemente si el algoritmo de Floyd-Warshall se puede adaptar en forma sencilla para la versión todos a todos del problema.
 \end{enumerate}
 
 
 % FIGURA
