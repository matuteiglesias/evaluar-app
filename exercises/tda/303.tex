
\item\Obligatorio Un \emph{grafo orientado} (ver Figura~\ref{fig:grafo orientado}) es un digrafo $D$ tal que al menos uno de $v \to w$ y $w \to v$ no es una arista de $D$, para todo $v,w \in V(D)$.  En otras palabras, un grafo orientado se obtiene a partir de un grafo no orientado dando una dirección a cada arista.  Demostrar en forma constructiva que para cada $n$ existe un único grafo orientado cuyos vértices tienen todos grados de salida distintos.  \textbf{Ayuda:} aprovechar el ejercicio anterior y observar que el absurdo no se produce para un único grafo orientado.

 % FIGURA
