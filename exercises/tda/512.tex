
 \item Adaptar el modelo del Ejercicio~\ref{ej:SRD} para el caso en que se reemplazan las inecuaciones por ecuaciones; en este caso, el sistema se describe con un grafo $G({\cal S})$ en lugar del digrafo $D({\cal S})$.  Luego, considerar el problema C del Torneo Argentino de Programación 2017\footnote{Autor: Nicolás Álvarez, Universidad Nacional del Sur}, cuyo texto se copia abajo.

 \begin{enumerate}[label=$\alph*$.,ref=$\alph*$]
  \item Modelar el problema con un SRD ${\cal S}$ que utilice únicamente ecuaciones.  En el mejor modelo que conocemos cada incógnita aparece una única vez en forma positiva y una única vez en forma negativa y, por lo tanto, cada componente de $G({\cal S})$ es un ciclo.
  \item Proponer un algoritmo de tiempo $O(n)$ para resolver el problema, donde $n$ es la cantidad de incógnitas en ${\cal S}$.
 \end{enumerate}

 \begin{quotation}
  Carolina tiene la costumbre de juntarse todas las tardes a tomar mate con sus amigos.  Como han vivido equivocados toda su vida, les gusta tomar mate dulce. Últimamente, se han preocupado por su ingesta calórica y han decidido probar un nuevo edulcorante cero calorías que salió al mercado: el Ingrediente Caramelizador de Productos Cebables (ICPC). El ICPC tiene la extraña propiedad de que al aplicarlo dura exactamente $K$ cebadas endulzando el mate y luego se evapora completamente.

 Carolina y sus amigos se ubican alrededor de una mesa circular, y se numeran del $0$ al $N - 1$ en el sentido de las agujas del reloj. Luego comienzan a tomar mate durante varias rondas. En cada ronda, ella ceba un mate para cada integrante, comenzando por la persona $0$ y continuando en orden ascendente hasta llegar a la persona $N - 1$. Por lo tanto, luego de que toma la persona $N - 1$ es nuevamente el turno de la persona 0. Carolina decide una cantidad fija entera y positiva $E_i$ de ICPC para agregar al mate antes de cebar a la persona $i$. La cantidad de ICPC que recibe cada persona en su mate será entonces la suma de lo agregado por la cebadora en las últimas $K$ cebadas. Formalmente, la cantidad de ICPC que recibe la persona $i$ a partir de la segunda ronda es
 
 
$$
  T_i = \sum_{d=0}^{K-1}E_{(i-d) \bmod N}
$$
 
 
 donde $x \bmod N$ es un entero entre $0$ y $N - 1$ que indica el resto de $x$ en la división entera por $N$.

 Por ejemplo, si la ronda constara de $N = 5$ amigos, la duración del edulcorante fuera de $K = 3$ cebadas y las cantidades de ICPC agregado fueran $E_0 = 10$, $E_1 = 4$, $E_2 = 0$, $E_3 = 2$ y $E_4 = 1$, entonces las cantidades de ICPC que recibirían los amigos serían $T_0 = 13$, $T_1 = 15$, $T_2 = 14$, $T_3 = 6$ y $T_4 = 3$.

 Carolina conoce muy bien los gustos de sus amigos y quisiera complacerlos a todos. Dado un arreglo $G_0$, $G_1$, \ldots, $G_{N-1}$ con las cantidades de edulcorante que quieren recibir los $N$ amigos, ustedes deben determinar si existe un arreglo $E_0$, $E_1$, \ldots, $E_{N-1}$ con las cantidades de ICPC a agregar antes de cebar a cada persona, tal que a partir de la segunda ronda todos los amigos estén satisfechos (esto es, $T_i = G_i$ para $i = 0, 1, \ldots, N - 1$).
 \end{quotation}
