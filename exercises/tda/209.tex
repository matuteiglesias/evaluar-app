
Dados dos arreglos de naturales, ambos ordenados de manera
creciente, se desea buscar, dada una posición $i$, el $i$-ésimo elemento de
la unión de ambos. Dicho de otra forma, el $i$-ésimo del resultado
de hacer merge ordenado entre ambos arreglos.
Notar que no es necesario hacer el merge completo.
Se puede asumir que cada natural aparece a lo sumo en uno de los arreglos,
y a lo sumo una vez.

\begin{enumerate}
\item[a)] Implementar la función
\begin{center}
iésimoMerge(\textbf{in} $A$: arreglo(nat), \textbf{in} $B$: arreglo(nat), \textbf{in} $i$: nat) $\to$ nat
\end{center}
que resuelve el problema planteado. 
La función debe ser de tiempo $O(\log^2 n)$, dónde
$n = \tam(A) = \tam(B)$.
\item[b)] Calcular y justificar la complejidad del algoritmo propuesto.
\item[c)] Intente resolver el mismo problema en tiempo $O(\log n)$ (este ítem
es bastante mas difícil, se incluye como desafío adicional).
\end{enumerate}

