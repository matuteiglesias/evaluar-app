
Suponga que se tiene un método \emph{potencia} que, dada un matriz cuadrada $A$ de orden $4 \times 4$ y un número $n$, computa la matriz $A^n$.  Dada una matriz cuadrada $A$ de orden $4 \times 4$ y un número natural $n$ que es potencia de $2$ (i.e., $n = 2^k$ para algun $k \geq 1$), desarrollar, utilizando la técnica de dividir y conquistar y el método {\it potencia}, un algoritmo que permita calcular \[ A^1 + A^2 + \ldots + A^n. \]   
Calcule el número de veces que el algoritmo propuesto aplica el método {\it potencia}.  Si no es estrictamente menor que $O(n)$, resuelva el ejercicio nuevamente.
