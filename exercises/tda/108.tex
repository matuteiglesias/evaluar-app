

\item Debemos cortar una vara de madera en varios lugares predeterminados. Sabemos que el costo de realizar un corte en una madera de longitud $\ell$ es $\ell$ (y luego de realizar ese corte quedarán 2 varas de longitudes que sumarán $\ell$).  Por ejemplo, si tenemos una vara de longitud 10 metros que debe ser cortada a los 2, 4 y 7 metros desde un extremo, entonces los cortes se pueden realizar, entre otras maneras, de las siguientes formas:
%\begin{itemize}
\item Primero cortar en la posición 2, después en la 4 y después en la 7. Esta resulta en un costo de 10 + 8 + 6 = 24 porque el primer corte se hizo en una vara de longitud 10 metros, el segundo en una de 8 metros y el último en una de 6 metros.
\item Cortar primero donde dice 4, después donde dice 2, y finalmente donde dice 7, con un costo de 10 + 4 + 6 = 20, que es menor.
%\end{itemize}
\par{Queremos encontrar el mínimo costo posible de cortar una vara de longitud $\ell$.}
\begin{enumerate}[label=$\alph*)$,ref=$\alph*)$]
\item Convencerse de que el mínimo costo de cortar una vara que abarca desde $i$ hasta $j$ con el conjunto $C$ de lugares de corte es $j-i$ mas el mínimo, para todo lugar de corte $c$ entre $i$ y $j$, de la suma entre el mínimo costo desde $i$ hasta $c$ y el mínimo costo desde $c$ hasta $j$.
\label{maderaFormRec1}
\item Escribir matemáticamente una formulación recursiva basada en \ref{maderaFormRec1}. Explicar su semántica e indicar cuáles serían los parámetros para resolver el problema.
\item Diseñar un algoritmo de PD y dar su complejidad temporal y espacial auxiliar. Comparar cómo resultaría un enfoque \textit{top-down} con uno \textit{bottom-up}.
\label{maderaComplejidad1}
\item Supongamos que se ordenan los elementos de $C$ en un vector $cortes$ y se agrega un $0$ al principio y un $\ell$ al final. Luego, se considera que el mínimo costo para cortar desde el $i$-ésimo punto de corte en $cortes$ hasta el $j$-ésimo punto de corte será el resultado buscado si $i = 1$ y $j = |C| + 2$.
\label{maderaFormRec2}
\begin{enumerate}[label=$\roman*)$,ref=$\roman*)$]
\item Escribir una formulación recursiva con dos parámetros que esté basada en \ref{maderaFormRec2} y explicar su semántica.
\item Diseñar un algoritmo de PD, dar su complejidad temporal y espacial auxiliar y compararlas con aquellas de \ref{maderaComplejidad1}. Comparar cómo resultaría un enfoque \textit{top-down} con uno \textit{bottom-up}.
\end{enumerate}
\end{enumerate}


