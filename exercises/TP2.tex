El horizonte de Murcia está creciendo muy rápidamente. Desde el siglo XV, solía estar dominado por el perfil de su Catedral Barroca. Pero hoy en día, nuevos rascacielos están surgiendo en la huerta murciana. 

Algunas personas dicen que si miras el horizonte de izquierda a derecha, puedes observar un perfil creciente; pero otras personas dicen que el perfil es decreciente. 

Mirando el horizonte de Murcia de izquierda a derecha, tenemos una serie de N edificios. Cada edificio tiene su propia altura y anchura. Tienes que descubrir si el horizonte está creciendo o decreciendo. 

Decimos que el horizonte está creciendo si la subsecuencia creciente más larga de edificios es mayor o igual que la subsecuencia decreciente más larga de edificios; en otro caso, decimos que está decreciendo. Una subsecuencia es un subconjunto de la secuencia original, en el mismo orden. La longitud de una subsecuencia de edificios es la suma de las anchuras de sus elementos. 

Por ejemplo, asumiendo que tenemos seis edificios de alturas: 10, 100, 50, 30, 80, 10; y anchuras: 50, 10, 10, 15, 20, 10; entonces tenemos una subsecuencia creciente de 3 edificios y longitud total 85, y una subsecuencia decreciente de 1 edificio y longitud total 50 (además, hay una subsecuencia decreciente de 4 edificios y longitud 45). Así, en este caso, decimos que el horizonte está creciendo. Puedes ver este ejemplo a continuación. 

Entrada 
La primera línea de la entrada contiene un entero indicando el número de casos de prueba. Para cada caso de prueba, la primera línea contiene un solo entero, N, indicando el número de edificios del horizonte. Luego, hay dos líneas, cada una con N enteros separados por espacios en blanco. La primera línea indica las alturas de los edificios, de izquierda a derecha. La segunda línea indica las anchuras de los edificios, también de izquierda a derecha. 

Salida 
Para cada caso de prueba, la salida debe contener una línea. Si el horizonte está creciendo, el formato será: 
Caso i. Creciente (A). Decreciente (B). 
Si el horizonte está decreciendo, el formato será: 
Caso i. Decreciente (B). Creciente (A). 
donde i es el número del caso de prueba correspondiente (comenzando con 1), A es la longitud de la subsecuencia creciente más larga, y B es la longitud de la subsecuencia decreciente más larga. 

