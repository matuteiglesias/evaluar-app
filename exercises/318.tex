
 \item Decimos que un digrafo (con \textit{loops}) tiene forma de $\rho$ cuando todos sus vértices tienen grado de salida igual a 1 (Figura~\ref{fig:grafos-rho}).
 \begin{enumerate}[label=$\alph*)$,ref=$\alph*)$]
  \item Demostrar en forma constructiva que si un digrafo es conexo y tiene forma de $\rho$ entonces tiene un único ciclo dirigido.  Notar que si $v \to v$ es un \textit{loop}, entonces $v, v$ es un ciclo.  Recordar que un digrafo es conexo cuando su grafo subyacente es conexo.
  \item Diseñar un algoritmo para encontrar todos los ciclos de un digrafo con forma de $\rho$ (no necesariamente conexo). \textbf{Ayuda:} notar que los vértices con grado de entrada $0$ no están en ciclos; luego, se pueden sacar iterativamente.  Demostrar que todos los vértices tienen grado de entrada $1$ en el grafo resultante y, por lo tanto, todos los vértices pertenecen a un ciclo.
 \end{enumerate}

 Consideremos un período de tiempo circular $[0, T]$ tal que llegado el momento $T$ se vuelve a contabilizar el tiempo 0 (e.g., un día, una semana, etc).  Dentro del tiempo $[0, T]$ se encuentran definidas $n$ actividades, la $i$-ésima de las cuales se desarrolla empezando en el instante $s_i$ y terminando en el instante $t_i$ (si $s_i > t_i$, entonces la actividad contiene el instante $0 = T$).

 En el problema de selección de actividades periódicas se busca determinar la máxima cantidad de actividades que un agente puede realizar rutinariamente, suponiendo que el agente es capaz de realizar una única actividad en cada instante.  Formalmente, el objetivo es determinar la máxima razón $x/y$ para una secuencia circular de actividades $A_1, \ldots, A_x$ que se realiza en $y$ períodos completos cuando $A_{i+1}$ se inicia lo antes posible una vez terminado $A_i$, para todo $1 \leq i \leq x$ (con $A_{x+1} = A_1$).  Considere la siguiente estrategia golosa para decidir qué actividad $j$ conviene elegir si se elige la actividad $i$: tomar $j$ como una actividad cuyo tiempo de finalización es el primero desde $t_i$ cuando $j$ se empieza después de terminar la actividad $i$, en un recorrido del tiempo en el sentido de las agujas del reloj.

 Definir el digrafo de actividades $D$ que tiene un vértice $i$ por cada actividad y que tiene un arco (arista) $i \to j$ cuando $j$ es la elección golosa que se toma si se elige $i$.

 \begin{enumerate}[resume*]
  \item Observar que $D$ es un digrafo con forma de $\rho$.
  \item \textbf{(Difícil)} Demostrar que algún ciclo de $D$ es una solución al problema de selección de actividades. \textbf{Ayuda:} demostrar por inducción en $i$ ($i \leq x$) que existe una solución $A_1, \ldots, A_x$ donde $A_{i+1}$ es la elección golosa para $A_i$, tomando $A_{x+1} = A_1$.
  \item Usando los resultados anteriores, dar un algoritmo para resolver el problema de selección de actividades, suponiendo que las actividades se describen usando un conjunto de pares $[s_i, t_i]$ donde $0 \leq s_i, t_i \leq T$, $s_i \neq t_i$ y $T = 2n$.
 \end{enumerate}
\end{enumerate}

% FIGURA
