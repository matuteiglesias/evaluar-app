
 \item\Obligatorio Se tiene una grilla con $m \times n$ posiciones, cada una de las cuales tiene un número entero en $[0, k)$, para un $k \in \mathbb{N}$ dado.  Dado un valor objetivo $w \in \mathbb{N}$ y una posición inicial $(x_1, y_1)$, que tiene un valor inicial $v_1$, queremos determinar la mínima cantidad de movimientos horizontales y verticales que transformen $v_1$ en $w$, teniendo en cuenta que el $i$-ésimo movimiento transforma a $v_i$ por $v_{i+1} = (v_i + z) \bmod k$, donde $z$ es el valor que se encuentra en la casilla de destino del movimiento.  Por ejemplo, para la siguiente grilla y $k=10$, se puede transformar $v_1 = 1$ en $w = 0$ con tres movimientos $1 \to 6 \to 4 \to 9$, aunque la solución óptima es vía el camino $1 \to 3 \to 6$.

 <table>
   <tr>
     <td>1</td>
     <td>3</td>
     <td>6</td>
   </tr>
   <tr>
     <td>6</td>
     <td>7</td>
     <td>4</td>
   </tr>
   <tr>
     <td>4</td>
     <td>9</td>
     <td>-3</td>
   </tr>
</table>


 Modelar este problema como un problema de grafos que se resuelva usando BFS en $O(kmn)$ tiempo.  


