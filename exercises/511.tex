
 
 \item Nuevamente tenemos a $n$ clientes de un supermercado $\{c_1, c_2, \ldots, c_n\}$ y queremos asignarle a cada uno una caja para hacer fila. Esta vez, las cajas están ordenadas en forma circular, numeradas de la $1$ a la $M$ y se encuentran separadas por pasillos. Entre la caja $M$ y la $1$ hay una valla que impide pasar de una a la otra. Durante el proceso de asignación algunos clientes se pelean entre sí y son separados por seguridad. Si dos clientes $c_i$ y $c_j$ se pelean, los guardias les dicen que tienen que ponerse en filas distintas que se encuentren separadas por al menos $K_{ij} > 0$ pasillos intermedios en ambos sentidos del círculo, para que no se vuelvan a pelear. Notar que cuando seguridad separa una pelea naturalmente hay un cliente que queda en un número de caja más bajo y el otro en un número de caja más alto. Con la restricción de no volver a acercarse y la valla entre las cajas $M$ y $1$ ese orden ya no puede cambiar. ¿Será posible asignarlos a todos?
 
 \begin{enumerate}[label=$\alph*$.,ref=$\alph*$]
  \item Modelar el problema utilizando un sistema de restricciones de diferencias.  Para el modelo, notar que sabemos qué clientes se pelearon.  Más aún, si $c_i$ y $c_j$ se pelearon, sabemos quién entre $c_i$ y $c_j$ quedó del lado de las cajas con menor numeración. En este escenario no hay restricciones por amistad.
  \item Proponer un algoritmo polinomial que lo resuelva.
  \item ¿Qué complejidad tiene el algoritmo propuesto?  Para la respuesta, tener en cuenta la cantidad $m_1$ de peleas.
 \end{enumerate}


