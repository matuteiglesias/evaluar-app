

\item Un camino $P$ de $v$ a $w$ en un grafo $G$ es \emph{geodésico} cuando la cantidad de aristas de $P$ es la mínima entre todos los caminos que unen a $v$ con $w$ (i.e., $P$ es un camino mínimo entre $v$ y $w$).  El \emph{cuadrado} de un grafo $G$ es el grafo $G^2$ que tiene los mismos vértices que $G$ y donde $vw$ son adyacentes si y solo sus vecindarios cerrados tienen algún vértice en común.
 \begin{enumerate}[label=$\alph*)$,ref=$\alph*)$]
  \item \textbf{(Difícil)} Demostrar, usando la técnica de reducción al absurdo, que si $G$ tiene un camino geodésico con la menos $4$ aristas, entonces $\overline{G}^2$ es un grafo completo.
  \item Demostrar, usando el inciso anterior y la técnica de reducción al absurdo, que si $G$ tiene un camino geodésico con al menos $3$ aristas, entonces $\overline{G}$ no tiene caminos geodésicos con mas de $3$ aristas.
 \end{enumerate}
