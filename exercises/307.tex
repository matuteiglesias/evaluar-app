

\item La \emph{union} $G \cup H$ de dos grafos $G$ y $H$ es el grafo con $V(G \cup H) = V(G) \cup V(H)$ y $E(G \cup H) = E(G) \cup E(H)$.  Es decir, $G \cup H$ se obtiene uniendo $G$ con $H$ sin agregar aristas.  Por otra parte, la \emph{junta} $G + H$ de $G$ y $H$ es el grafo que se obtiene de $G \cup H$ agregando todas las aristas $vw$ posibles entre un vértice $v \in V$ y otro vértice $w \in V(H)$.  Decimos que $G$ es un \emph{grafo unión} (resp.\ \emph{junta}) si existen $G_1$ y $G_2$ tales que $G = G_1 \cup G_2$ (resp.\ $G = G_1 + G_2$).
\begin{enumerate}[label=$\alph*)$,ref=$\alph*)$]
\item Demostrar en forma directa que $G$ es un grafo unión si y solo si $G$ es disconexo.
\item Demostrar en forma directa que $G$ es un grafo junta si y sólo si $\overline{G}$ es un grafo unión.
\item Concluir que $G$ es un grafo junta si y sólo si $\overline{G}$ es disconexo.
\end{enumerate}

