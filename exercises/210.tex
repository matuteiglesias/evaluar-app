
Se tienen dos arreglos de $n$ naturales $A$ y $B$. $A$ está ordenado
de manera creciente y $B$ está ordenado de manera decreciente. Ningún valor
aparece mas de una vez en el mismo arreglo.
Para cada posición $i$ consideramos la diferencia absoluta entre los valores de
ambos arreglos $|A[i]-B[i]|$. Se desea buscar el mínimo valor posible de 
dicha cuenta. Por ejemplo, si los arreglos son $A = [1,2,3,4]$ y $B = [6,4,2,1]$
los valores de las diferencias son $5,2,1,3$ y el resultado es $1$.


\begin{enumerate}
\item[a)] Implementar la función
\begin{center}
minDif(\textbf{in} $A$: arreglo(nat), \textbf{in} $B$: arreglo(nat)) $\to$ nat
\end{center}
que resuelve el problema planteado. 
La función debe ser de tiempo $O(\log n)$, dónde $n = \tam(A) = \tam(B)$.
\item[b)] Calcular y justificar la complejidad del algoritmo propuesto. 
\end{enumerate}
