En la primera mañana de cada verano, cuando el primer rayo de sol irrumpe en el bosque de robles, Jayjay, la ardilla voladora, rápidamente se trepa hasta la cima de un roble en el bosque. Desde allí, comienza su descenso hacia el suelo, intentando recoger tantas bellotas de los árboles en su camino hacia abajo. Siendo una ardilla voladora, Jayjay puede elegir, en cualquier momento, descender por el tronco del árbol o volar de un árbol a otro en su viaje descendente. Sin embargo, pierde f pies de altura cada vez que vuela de un árbol a otro. 

Supongamos que el bosque tiene t robles, y todos los árboles tienen la misma altura de h pies. Dada la altura de cada bellota en cada árbol, escribe un programa para calcular el número máximo posible de bellotas que Jayjay puede recolectar eligiendo un árbol para escalar y descender como se describe. 

La Figura 2 muestra un ejemplo de t = 3 robles con tres, seis y cinco bellotas, respectivamente. Los círculos blancos y la línea gris indican un camino para que Jayjay recolecte el número máximo posible de ocho bellotas, asumiendo que la altura que pierde por cada vuelo es f = 2. 

Entrada 
La entrada consiste en una línea que contiene el número c de conjuntos de datos, seguido por c conjuntos de datos, seguido por una línea que contiene el número ‘0’. 

La primera línea de cada conjunto de datos contiene tres enteros, t, h, f, separados por un espacio en blanco. El primer entero t es el número de robles en el bosque. El segundo entero h es la altura (en pies) de todos los robles. El tercer entero, f, es la altura (en pies) que Jayjay pierde cada vez que vuela de un árbol a otro. Puedes asumir que 1 ≤ t, h ≤ 2000, y 1 ≤ f ≤ 500. 

La primera línea de cada conjunto de datos es seguida por t líneas. La i-ésima línea especifica la altura de cada bellota en el i-ésimo árbol. La línea comienza con un entero no negativo a que especifica cuántas bellotas tiene el i-ésimo árbol. Cada uno de los siguientes enteros a indica que una bellota está a la altura n en el i-ésimo árbol. Los enteros positivos en cada línea están ordenados en orden ascendente, y se permiten repeticiones. Por lo tanto, puede haber más de una bellota a la misma altura en el mismo árbol. Puedes asumir que 0 ≤ a ≤ 2000, para cada i. 

Salida 
La salida consiste en una línea para cada conjunto de datos. La c-ésima línea contiene un solo entero, que es el número máximo posible de bellotas que Jayjay puede recolectar en un solo descenso para el conjunto de datos c. 

Nota: El conjunto de datos a continuación y el camino de Jayjay para recolectar el número máximo de 8 bellotas se muestran en la Figura 2.
