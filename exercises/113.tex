
\item Se arrojan simultáneamente $n$ dados, cada uno con $k$ caras numeradas de $1$ a $k$. Queremos calcular todas las maneras posibles de conseguir la suma total $s \in \mathbb{N}$ con una sola tirada. Tomamos dos variantes de este problema.

\begin{enumerate}[label=$(\Alph*)$,ref=$(\Alph*)$]
\item Consideramos que los dados son \textbf{distinguibles}, es decir que si $n = 3$ y $k = 4$, entonces existen $10$ posibilidades que suman $s = 6$:
\begin{enumerate}
\item $4$ posibilidades en las que el primer dado vale $1$
\item $3$ posibilidades en las que el primer dado vale $2$
\item $2$ posibilidades en las que el primer dado vale $3$
\item Una posibilidad en la que el primer dado vale $4$
\end{enumerate}\label{ejdados:distinguibles}
\item Consideramos que los dados son \textbf{indistinguibles}, es decir que si $n = 3$ y $k = 4$, entonces existen $3$ posibilidades que suman $s = 6$:
\begin{enumerate}
\item Un dado vale $4$, los otros dos valen $1$
\item Un dado vale $3$, otro $2$ y otro $1$
\item Todos los dados valen $2$
\end{enumerate}\label{ejdados:indistinguibles}
\end{enumerate}

\begin{enumerate}[label=$\alph*)$,ref=$\alph*)$]
  \item Definir en forma recursiva la función $f \colon \mathbb{N}^2 \to \mathbb{N}$ tal que $f(n, s)$ devuelve la respuesta para el escenario~\ref{ejdados:distinguibles} (fijado $k$).
  \item Definir en forma recursiva la función $g \colon \mathbb{N}^3 \to \mathbb{N}$ tal que $f(n, s, k)$ devuelve la respuesta para el escenario~\ref{ejdados:indistinguibles}.
  \item Demostrar que $f$ y $g$ poseen la propiedad de superposición de subproblemas.
  \item Definir algoritmos \textit{top-down} para calcular $f(n, s)$ y $g(n, s, k)$ indicando claramente las estructuras de datos utilizadas y la complejidad resultante.
  \item Escribir el (pseudo-)código de los algoritmos top-down resultantes.
\end{enumerate}
 \textbf{Nota:} Una solución correcta de este ejercicio debería indicar cómo se computa tanto $f(n,s)$ como $g(n,s,k)$ en tiempo $O(nk\min\{s, nk\})$.


