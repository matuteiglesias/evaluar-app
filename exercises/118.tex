
\item Se define la función $mex: \mathcal{P}(\mathbb{N}) \to \mathbb{N}$ como

$$
mex(X) = \min\{j : j\in \mathbb{N} \wedge j \notin X\}
$$

Intuitivamente, $mex$ devuelve, dado un conjunto $X$, el menor número natural que no está en $x$. Por ejemplo, $mex(\{0,1,2\}) = 3$, $mex(\{0, 1, 3\}) = 2$ y $mex(\{1,2,3, \ldots\}) = 0$.

Dado un vector de número $a_1 \ldots a_n$ queremos encontrar la permutación $b_1 \ldots b_n$ de los mismos que maximize

$$
\sum_{i=1}^n mex(\{b_1 \ldots b_i\})
$$

Por ejemplo, si el vector es $\{3, 0, 1\}$ podemos ver que la mejor permutación es $\{0, 1, 3\}$, que alcanza un valor de

$$
mex(\{0\}) + mex(\{0, 1\}) + mex(\{0,1,3\}) = 1+2+2 = 5
$$

\begin{enumerate}
    \item Proponer un algoritmo \textit{greedy} que resuelva el problema y demostrar su correctitud. \textbf{Ayuda}: ¿Cuál el máximo valor que puede tomar $mex(X)$ si $X$ tiene $n$ elementos? Si $X \subseteq Y$, ¿Qué pasa con los valores $mex(X)$ y $mex(Y)$?

    \item Dar una implementación del algoritmo del inciso anterior con complejidad temporal $O(n)$.
\end{enumerate}

