

 \item\Obligatorio Discutir (brevemente) las ventajas y desventajas en cuanto a la complejidad temporal y espacial de las siguientes implementaciones de un conjunto de vecindarios para un grafo $G$, de acuerdo a las siguientes operaciones:
 \begin{description}
  \item [Operaciones] ~
  \begin{enumerate}
   \item Inicializar la estructura a partir de un conjunto de aristas de $G$.
   \item Determinar si dos vértices $v$ y $w$ son adyacentes.
   \item Recorrer y/o procesar el vecindario $N(v)$ de un vértice $v$ dado.
   \item Insertar un vértice $v$ con su conjunto de vecinos $N(v)$.
   \item Insertar una arista $vw$.
   \item Remover un vértice $v$ con todas sus adyacencias.
   \item Remover una arista $vw$.
   \item Mantener un orden de $N(v)$ de acuerdo a algún invariante que permita recorrer cada vecindario en un orden dado.
  \end{enumerate}
  \item [Estructuras de datos] ~
  \begin{enumerate}
   \item $N$ se representa con una secuencia (vector o lista) que en cada posición $v$ tiene el conjunto $N(v)$ implementado sobre una secuencia (lista o vector).  Cada vértice es una estructura que tiene un índice para acceder en $O(1)$ a $N(v)$.  Esta representación se conoce comúnmente como \emph{lista de adyacencias}.
   \item ídem anterior, pero cada $w \in N(v)$ se almacena junto con un índice a la posición que ocupa $v$ en $N(w)$.  Esta representación también se conoce como \emph{lista de adyacencias}, pero tiene información para implementar operaciones dinámicas.
   \item $N(v)$ se representa con un vector que en cada posición $i$ tiene un vector booleano $A_i$ con $|V(G)|$ posiciones tal que $A_i[j]$ es verdadero si y solo si $i$ es adyacente a $j$.  Esta representación se conoce comúnmente como \emph{matriz de adyacencias}.
   \item $N(v)$ se representa con un vector que en cada posición tiene el conjunto $N(v)$ implementado con una tabla de hash.  Esta representación es un mix entre las representaciones clásicas de matriz de adyacencias y lista de adyacencias.
  \end{enumerate}
 \end{description}



