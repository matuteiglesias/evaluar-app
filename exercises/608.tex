
 \item Dado un ordenamiento $v_1, \ldots, v_n$ de los vértices de un digrafo $D$, se define la \emph{secuencia digráfica} de $D$ como $(d^-(v_1), d^+(v_1)), \ldots, (d^-(v_n), d^+(v_n))$.
 Dada una secuencia de pares $d$, el problema de realización de $d$ consiste en encontrar un digrafo $D$ cuya secuencia digráfica sea $d$.
 \begin{enumerate}[label=$\alph*)$,ref=$\alph*)$]
  \item Modelar el problema de realización como un problema de flujo.
  \item Dar una interpretación a cada unidad de flujo y cada restricción de capacidad.
  \item Demostrar que el modelo es correcto.
  \item Determinar la complejidad de resolver el modelo resultante con el algoritmo de Edmonds y Karp.  La cota debe estar expresada en función de $n$ y debe ser lo suficientemente ajustada.
 \end{enumerate}
  

