

  \item En muchas aplicaciones se necesita encontrar caminos de \emph{peso multiplicativo} mínimo en un digrafo $D$ pesado con una función positiva $c \colon E(G) \to \mathbb{R}_{> 1}$.  Formalmente, el peso multiplicativo de un camino $v_1, \ldots, v_k$ es la multiplicatoria de los pesos de sus aristas.  Este tipo de caminos se buscan, por ejemplo, cuando los pesos de las aristas representan probabilidades\footnote{No se incluye un ejercicio en particular de este tema debido a que Probabilidad y Estadística no es correlativa con AED3.} de eventos independientes y se quiere encontrar una sucesión de eventos con probabilidad máxima/mínima.  Modelar el problema de camino de peso multiplicativo mínimo como un problema de camino mínimo.  \textbf{Demostrar} que el algoritmo es correcto.  \textbf{Ayuda:} transformar el peso de cada arista usando una operación conocida que sea creciente y transforme cualquier multiplicatoria en una sumatoria.

