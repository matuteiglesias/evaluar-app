

 \item En el pueblo de \emph{Asignasonia} las fiestas de casamiento son muy peculiares y extrañamente frecuentes.  Las invitaciones a la fiesta nunca son personales sino familiares: cada persona invitada asiste siempre con todes sus familiares solteres, a quienes se les reservan mesas especiales de solteres.  Además, hay una regla no escrita que establece un límite $c_{ij}$ a la cantidad de solteres de la familia $i$ que pueden sentarse en la mesa $j$.  Esta forma de festejar es la que, aparentemente, aumenta la cantidad de casamientos futuros.  Desafortunadamente, el esfuerzo que implica mantener viva esta tradición está llevando a que varias parejas eviten el compromiso marital.  Es por esto que la intendencia de Asignasonia requiere un algoritmo que resuelva el problema de asignación de les solteres a sus mesas.  
 \begin{enumerate}[label=$\alph*)$,ref=$\alph*)$]
  \item Proponer un modelo de flujo que dados los conjuntos $F = \{f_1, \ldots, f_{|F|}\}$, $M = \{m_1, \ldots, m_{|M|}\}$ y $C = \{c_{ij} \mid 1 \leq i \leq |F|, 1 \leq j \leq |M|\}$ determine una asignación que respete las tradiciones sabiendo que:
  %\begin{itemize}
   \item la familia $i$ esta formada por $f_i$ personas solteres,
   \item la mesa $j$ tiene $m_j$ lugares disponibles para solteres, y
   \item en la mesa $j$ solo pueden sentarse $c_{ij}$ solteres de la familia $i$.  
  %\end{itemize}
  \item Dar una interpretación a cada unidad de flujo y cada restricción de capacidad.
  \item Determinar la complejidad de resolver el modelo resultante con el algoritmo de Edmonds y Karp.
 \end{enumerate}


