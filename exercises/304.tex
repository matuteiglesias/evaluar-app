


\item Un vértice $v$ de un grafo $G$ es un \emph{punto de articulación} si $G-v$ tiene más componentes conexas que $G$. Por otro lado, un grafo es \emph{biconexo} si es conexo y no tiene puntos de articulación.

 \begin{enumerate}[label=$\alph*)$,ref=$\alph*)$]
    \item\Obligatorio Demostrar, usando inducción en la cantidad de vértices, que todo grafo de $n$ vértices que tiene más de $(n-1)(n-2)/2$ aristas es conexo. Opcionalmente, puede demostrar la misma propiedad usando otras técnicas de demostración.

    \item Demostrar por medio de una reducción al absurdo que todo grafo de $n$ vértices que tenga al menos $2+(n-1)(n-2)/2$ aristas es biconexo.

    \item ¿Se pueden dar cotas mejores que funcionen a partir de algún $n_0$? Es decir, ¿existe $c(n) < 1+(n-1)(n-2)/2$ (resp.\ $c(n) < 2+(n-1)(n-2)/2$) tal que todo grafo de $n \geq n_0$ vértices que tenga al menos $c(n)$ aristas sea conexo (resp.\ biconexo)?
 \end{enumerate}

