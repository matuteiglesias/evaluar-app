

 \item Para cada una de las siguientes sentencias sobre el problema de flujo máximo en una red $N$: demostrar que es verdadera o dar un contraejemplo.
 \begin{enumerate}[label=$\alph*)$,ref=$\alph*)$]
  \item Si la capacidad de cada arista de $N$ es par, entonces el valor del flujo máximo es par.
  \item Si la capacidad de cada arista de $N$ es par, entonces existe un flujo máximo en el cual el flujo sobre cada arista de $N$ es par.
  \item Si la capacidad de cada arista de $N$ es impar, entonces el valor del flujo máximo es impar.
  \item Si la capacidad de cada arista de $N$ es impar, entonces existe un flujo máximo en el cual el flujo sobre cada arista de $N$ es impar.
  \item Si todas las aristas de $N$ tienen capacidades racionales, entonces el flujo máximo es racional.
 \end{enumerate}


